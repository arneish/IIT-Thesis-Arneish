\vspace{-0.10in}
\section{Related Work}
\label{sec:related}
%
We categorize related work as follows.

\spara{$\bullet$ Frequent subgraphs mining.}
To mine a set of graphs, efficient frequent subgraph mining algorithms were proposed,
e.g., AGM \cite{IWM00}, FSG \cite{KK01}, gSpan \cite{YH02}, PathJoin \cite{VGS02}, MoFa \cite{BB02},
FFSM \cite{HWP03}, GASTON \cite{NK04}, SPIN \cite{HWPY04}, etc. Techniques were also developed
to mine maximal \cite{HWPY04,MMF10}, closed \cite{YH03}, discriminative \cite{YCHY08,RHS13},
statistically significant \cite{HasanZ09,JinYW10}, and representative subgraph patterns \cite{HasanCSBZ07,ZYL09}.
These methods adopt subgraph isomorphism as a way to count the support of graph patterns in multiple graphs. 
For a survey, we refer to \cite{KhanR17,CYH10}.

In the area of mining single massive graphs, \cite{KK04,ChenHLN06,FB07,BN08} developed techniques to calculate the support of graph patterns. 
The state-of-the-art technique for mining frequent subgraphs
in a single-large graph is GRAMI \cite{EASK14}. GRAMI uses MNI \cite{BN08} as subgraph frequency, and models the problem of
subgraph frequency evaluation as a constraint satisfaction problem. Algorithms for statistically significant graph patterns \cite{AroraSB14}
and discriminative graph patterns \cite{DangSBYH14,DangYBS15} over a single graph were also designed.

As noted earlier, correlated subgraphs are different from frequent subgraphs due to the
flexibility in which the constituent subgraph instances are connected. Moreover,
correlation computation between two subgraph patterns require enumerating
and finding distances between every pair of subgraph instances of
both these patterns, thereby making our problem more
memory intensive and computationally demanding.

\vspace{-0.05in}
\spara{$\bullet$ Approximate subgraphs mining.}
To tolerate certain structural and label differences in two graphs,
approximate pattern mining frameworks were developed in \cite{MendozaAM12,ChenYZH07,AnchuriZBGS13},
which can find patterns that are missed by exact mining algorithms. 
%Approximation for fast support computation was investigated in \cite{IyerL0VBS18}. 
Proximity patterns \cite{KYW10} were introduced to mine the top-$k$ set of node labels that co-occur frequently
in neighborhoods. Correlations between node labels and dense graph structures were identified in
\cite{GWZSY11,SMZ12}. In the {\sf{CSM}} problem, while we allow certain flexibility in terms of how the
constituent subgraph instances are connected, we still maintain fixed structures for subgraph instances.
Hence, {\sf{CSM}} is different from existing works on approximate subgraph mining.


\vspace{-0.05in}
\spara{$\bullet$ Correlation mining in graph databases.}
All prior works on correlated graph mining \cite{KCN08,KCY09,KeCY09,LatsiouP11,ZouCL09,KeCN07} considered
graph databases consisting of multiple graphs. In particular, \cite{ZouCL09,KCN08,KeCY09,KeCN07}
developed efficient algorithms for searching both the top-$k$ and threshold-based ``correlative'' subgraphs in the
database, which share {\em similar occurrence distributions with a given query graph}.
The incremental and streaming versions of the top-$k$ correlative subgraph search problem
were studied in \cite{LatsiouP11} and \cite{PanZ12}, respectively. Moreover, Ke et al. \cite{KCY09}
designed mining algorithms for automatically finding the top-$k$ frequent correlated subgraph pairs,
where two subgraphs are correlated if they share similar occurrence distributions
in the graph database. Our problem is significantly different and more complex than these
existing works: {\bf (1)} We consider mining over a single, large graph, while these works
consider searching and mining in a graph database having several small and medium-scale graphs.
{\bf (2)} Their ``correlation'' measures simple co-occurrence, i.e., if two
constituent subgraphs occur in the same set of graphs from the graph database.
In our ``correlation'' computation, we need to enumerate every pair of instances of
both these subgraphs in a single, large graph, and then verify their pairwise distances.
Hence, our {\sf{CSM}} problem is computationally more challenging.

\vspace{-0.05in}
\spara{$\bullet$ Correlation mining in other domains.}
Correlation mining has drawn extensive attention in diverse applications
due to its advantages in uncovering underlying dependencies, for example,
in market transactions \cite{XSTK04,ZX08,LeeKCH03}, sequence databases \cite{LinJDH12},
sequences of sets \cite{Benson0T18}, quantitative databases \cite{KeCN08}, time series data \cite{MueenNL10,HoPVHB19},
and even in spatial domain \cite{ChanLYW19}. To the best of our knowledge,
our work is the first application of correlation mining in a single, large graph.
