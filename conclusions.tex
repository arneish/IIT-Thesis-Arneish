\vspace{-0.10in}
\section{Conclusions}
\label{sec:conclusions}

A large body of work exists on mining recurring structural patterns among a group of nodes in the form of frequent subgraphs. However, \textit{can we mine recurring patterns among the frequent subgraphs itself?}  In this paper, we answer this question by mining correlated pairs of frequent subgraphs. Unlike frequent subgraphs mining, we not only need to identify if a subgraph is frequent, but also enumerate, maintain, and compute distances among \emph{all} instances of all frequent subgraphs. Managing instances imposes a severe scalability challenge on both computation and storage. We tackle this challenge by designing a data structure called \emph{Replica}, which stores all instances in a compact manner. Furthermore, Replica allows us to design a near-optimal approximation scheme to enumerate and identify instances in a highly efficient manner. Through extensive evaluation across a series of real datasets, we demonstrate that the proposed mining algorithm \emph{CSM} scales to million-sized networks, imparts up to $5$ orders of magnitude speed-up over baseline techniques, and discovers patterns that existing techniques fail to reveal. Overall, our work initiates a new line of research by mining higher-level patterns from the pattern space itself.

For future work, we propose to mine arbitrary-sized groups of correlated subgraphs instead of being restricted to pairs.
